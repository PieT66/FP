\documentclass[bibliography=totoc,captions=tableheading,titlepage=firstiscover, parskip=half, ]{scrartcl}

%\usepackage{scrhack}

\usepackage[aux]{rerunfilecheck}

\usepackage{polyglossia}
\setmainlanguage{german}

\usepackage{amsmath}
\usepackage{amssymb}
\usepackage{mathtools}

\usepackage{fontspec}
\usepackage[math-style=ISO,bold-style=ISO,sans-style=italic,nabla=upright,partial=upright,warnings-off={mathtools-colon,mathtools-overbracket,},]{unicode-math}
\setmathfont{Latin Modern Math}
\setmathfont{XITS Math}[range={scr, bfscr}]
\setmathfont{XITS Math}[range={cal, bfcal}, StylisticSet=1]

\usepackage[locale=DE,separate-uncertainty=true,per-mode=reciprocal,output-decimal-marker={,},]{siunitx}

%\usepackage[version=4,math-greek=defauft,text-greek=default,]{mhchem}

\usepackage[autostyle]{csquotes}

\usepackage{xfrac}

\usepackage{float}
\floatplacement{figure}{htbp}
\floatplacement{table}{htbp}

\usepackage[section,below,]{placeins}

\usepackage{pdflscape}

\usepackage[labelfont=bf,font=small,width=0.9\textwidth,]{caption}
\usepackage{subcaption}

\usepackage{graphicx}
\usepackage{grffile}

\usepackage{booktabs}

\usepackage{mleftright} % ersetze im Code \left durch \mleft und \right durch \mright, sofern kein Space gewünscht wird bspw. bei Operatoren.

\usepackage{microtype}

\usepackage[backend=biber,]{biblatex}
\addbibresource{lit.bib}
\addbibresource{programme.bib}

\usepackage[unicode,pdfusetitle,pdfcreator={},pdfproducer={},]{hyperref}
\usepackage{bookmark}

\usepackage[shortcuts]{extdash}

\usepackage[math]{blindtext}

\usepackage{microtype}

%\usepackage{showframe}
%\usepackage{lua-visual-debug}

\DeclarePairedDelimiter{\bra}{\langle}{\rvert}
\DeclarePairedDelimiter{\ket}{\rvert}{\rangle}
\DeclarePairedDelimiterX{\braket}[2]{\langle}{\rangle}{#1 \delimsize\vert #2}

\usepackage{pdflscape}

\usepackage{expl3}
\usepackage{xparse}

\ExplSyntaxOn

\NewDocumentCommand \dif {m}
{
\mathinner{\mathup{d} #1}
}

\let\vaccent=\v
\RenewDocumentCommand \v {}
{
  \TextOrMath{\vaccent}{\mathbf}
}

\ExplSyntaxOff

\usepackage{extdash} %anstelle von Bindestrichen "-", bitte "\-/" benutzen! Bei verbotenen Umbrüchen bitte \=/ benutzen, bspw. "$x$\=/Achse"

\author{
  Alexander Froch%
  \texorpdfstring{
    \\
    \href{mailto:Alexander.Froch@tu-dortmund.de}{Alexander.Froch@tu-Dortmund.de}
  }{}%
  \texorpdfstring{\and}{, }
  Stefan Abbing%
  \texorpdfstring{
    \\
    \href{mailto:Stefan.Abbing@tu-dortmund.de}{Stefan.Abbing@tu-Dortmund.de}
  }{}%
}
\publishers{TU Dortmund – Fakultät Physik}
